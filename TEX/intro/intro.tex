For thousand of years humankind have looked up to the beyond and wondered. Specifically our species has wondered about the motion of our solar system. Finally after Newton the mystery was solved. Newton developed the gravitational law, which made it possible to predict the motion of the planets. A couple of centuries later, Einstein came with the theory of general relativity and made a small refinement to the law of motion that Newton proposed. 
\\
\\
These laws are not enough to solve the motion of the planets. From the laws one can derive differential equations for the motion, which are not trivial or even possible to solve analytically. This is where computational methods are useful. With the tools developed in computtational physics we can make a prediction to the motion of the planets in our solar system.\footnote{\href{http://www.uio.no/studier/emner/matnat/fys/FYS3150/h17/index.html}{\color{blue}{Semester page for FYS3150 - Autumn 2017}}.} And because of our assignment we kind of have to do this to pass the course.\cite{project3}
\\
\\
In this project we will make an object oriented code to solve the solar system with the Forward Eulers method and the Verlet-Velocity method. Both method will be derived, discussed, implemented and benchmarked for the Earth-Sun system. The Verlect-Velocity algorithm proved to be vastly superior in terms of precision and therefore used in the further calculations. The Verlet-Velocity method was then used on the three-body system with varying mass and finally for our full solar system (+ pluto).
\\
\\
Finally an analysis on the motion of Mercurys perihelion as part of a two body system with the sun is conducted. In contrast to the other plantes in the solar system, the precession of Mercurys perihelion can not be explained by classical predictions alone. This proved to be a major thorn in late 19th centurys astrophysicists side as the observational data perplexingly did not match up with, what was considered, the indefectibility of Newtons gravitational law. The historical significance of this phenomenon in addition to its paradigm-dualistic nature makes it particularly interesting. Our experiments show that we get an accurate prediction of the perihelion precession when relativistic effects are considered, which contribute with a missing angular deviation of $\theta$ = $43''$.